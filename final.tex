\documentclass{article}
\title{Abschlussaufgabe V - Partikelfilter in ARNL}
\author{Nils Petersohn, Matrikel: 20022749}
\date{3. Feb. 2010}
\begin{document}
\maketitle
%\begin{abstract}
%\end{abstract}



\section{Aufgabenstellung}
Es soll die monte-carlo-lokalisierung in Arnl untersucht werden. Durch Experimente und Recherche soll der genaue Ablauf des Partikelfilters und die Wirkung der Parameter dargestellt werden. Ausserdem soll geklaert werden wie die lokalisierung in eigenen programmen genutzt werden kann.


\section{Ueberblick}


\section{Theoretischer Hintergrund}\label{previous work}

\subsection{ARIA Architektur}
\subsection{ARNL}
\subsection{ArNetworking}





\bibliographystyle{abbrv}
\bibliography{main}

\end{document}